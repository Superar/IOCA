\chapter{Aleatoriedade}

Como usar escolhas aleatórias para melhorar os algoritmos. Alguns dos métodos mais avançados (algoritmos de aproximação e meta-heurísticas) costumam usar a aleatoriedade (Ver \hyperref[sec:metaheuristicas]{última aula da Semana 3}).

Trocar decisões \textbf{arbitrárias} (\underline{determinísticas}) por escolhas \textbf{aleatórias}.

\begin{itemize}
	\item Escolha do vértice inicial
	\item Ordem de se percorrer uma lista
	\item Construir uma solução viável inicial
	\item Escolher o vizinho em uma busca local
	\item Critério de desempate
\end{itemize}

Assumindo que exista um algoritmo aleatorizado \lstinline{AlgAleat}, o algoritmo genérico (para um problema de maximização) é:

\begin{algorithm}
	\SetAlgoLined
	\SetKwFunction{AlgAleat}{AlgAleat}
	\SetKwFunction{MultiStart}{MultiStart}
	\Fn{\MultiStart{$I, k$}}{
		best\_sol $\gets -\infty$\;
		\Para{$i \gets 1$ até $k$}{
			curr\_sol $\gets$ \AlgAleat{$I$}\;
			\Se{curr\_sol $>$ best\_sol}{
				best\_sol $\gets$ curr\_sol\;
			}
		}
		\Retorna{best\_sol}
	}
\end{algorithm}

É possível também fazer um algoritmo sensível à melhoria. Por exemplo, fazer parar o algoritmo quando passar $k$ iterações \textbf{sem} melhoria ao invés de ser fixo por $k$ interações. Continua procurando enquanto melhorar.

\section{Problema do Escalonamento}

Usaremos o algoritmo \lstinline{EscalonaGusloso2} visto na \hyperref[chp:semana2]{Semana 2}. No exemplo, encontramos um caso em que é melhor não ordenar as tarefas (ao ordenar o resultado é determinístico), ou seja, podemos criar um algoritmo aleatorizado que pode chegar a um caso não ordenado que me dê um melhor resultado.

\begin{example}
	$m=2$

	\tikzset{tempo/.style 2 args={rectangle split, rectangle split horizontal, draw=#2, rectangle split parts=#1, fill=#2!20}}

	\begin{center}
		\begin{tikzpicture}[node distance=.25cm]
			\node[tempo={3}{orange}] (t1) at (0, 0) {};
			\node[tempo={2}{blue}] (t2) [right=of t1] {};
			\node[tempo={2}{red}] (t3) [right=of t2] {};
			\node[tempo={3}{green}] (t4) [right=of t3] {};
			\node[tempo={2}{violet}] (t5) [right=of t4] {};
		\end{tikzpicture}
	\end{center}

	\begin{multicols}{2}
		Ordenando:

		\begin{tikzpicture}[node distance=.25cm]
			\node[tempo={3}{orange}] (t1) at (0, 0) {};
			\node[tempo={3}{green}] (t2) [right=of t1] {};
			\node[tempo={2}{blue}] (t3) [right=of t2] {};
			\node[tempo={2}{red}] (t4) [right=of t3] {};
			\node[tempo={2}{violet}] (t5) [right=of t4] {};
		\end{tikzpicture}

		\vspace{\baselineskip}

		\begin{tabular}{l|l}
			$M_1$ & \tikzmark{8m11} \\
			$M_2$ & \tikzmark{8m21} \\
		\end{tabular}

		\begin{tikzpicture}[remember picture, , tempo/.append style={anchor=west, overlay}, node distance=1pt]
			\node[tempo={3}{orange}] (t1) at (pic cs:8m11) {};
			\node[tempo={3}{green}] (t2) at (pic cs:8m21) {};
			\node[tempo={2}{blue}] (t3) [right=of t1] {};
			\node[tempo={2}{red}] (t4) [right=of t2] {};
			\node[tempo={2}{violet}] (t5) [right=of t3] {};
		\end{tikzpicture}

		\begin{center}
			\textit{makespan} = $7$
		\end{center}

		\columnbreak

		Sem ordenar:

		\vspace{\baselineskip}
		\vspace{\baselineskip}

		\begin{tabular}{l|l}
			$M_1$ & \tikzmark{8m12} \\
			$M_2$ & \tikzmark{8m22} \\
		\end{tabular}

		\begin{tikzpicture}[remember picture, , tempo/.append style={anchor=west, overlay}, node distance=1pt]
			\node[tempo={3}{orange}] (t1) at (pic cs:8m12) {};
			\node[tempo={2}{blue}] (t2) at (pic cs:8m22) {};
			\node[tempo={2}{red}] (t3) [right=of t2] {};
			\node[tempo={3}{green}] (t4) [right=of t1] {};
			\node[tempo={2}{violet}] (t5) [right=of t3] {};
		\end{tikzpicture}

		\begin{center}
			\textit{makespan} = $6$
		\end{center}
	\end{multicols}
\end{example}

\begin{algorithm}
	\SetAlgoLined
	\SetKwFunction{EscalonaAleat}{EscalonaAleat}
	\Fn{\EscalonaAleat{$n, t, m$}}{
		\lPara{$j \gets 1$ até $m$}{$M_j \gets \emptyset$}
		Escolha uma permutação aleatória dos itens e renomeie de acordo\;
		\Para{$i \gets 1$ até $n$}{
			seja $j$ uma máquina em que $\sum_{i \in M_j}t_i$ é mínimo\;
			$M_j \gets M_j \bigcup \{i\}$\;
		}
		\Retorna{$\max_{j=1,\dots,m}\sum_{i\in M_j}t_i$}
	}
\end{algorithm}

\subsection{Heurística de Busca Local}

Podemos aplicar também a aleatoriedade para heurísticas de busca local. Como exemplo, temos a função \lstinline{EscalonaBuscaLocal} da \hyperref[chp:semana3]{Semana 3}.

\begin{algorithm}
	\SetAlgoLined

	\Fn{\EscalonaUm{n, t, m}}{
		$\mathcal{M} \gets $ um escalonamento inicial\;
		\Enqto{houver um item $i'$ na máquina mais carregada $j'$ e uma máquina $j$ tal que $l(j) + t_{i'} < l(j')$}{
			$M_{j'} \gets M_{j'} \setminus \left\{ i'\right\}$\;
			$M_j \gets M_j \cup \left\{ i'\right\}$\;
		}
		\Retorna $\max_{j=1, \dots , m}\sum_{i \in M_j}t_i$
	}
\end{algorithm}

Podemos ter mais de uma máquina $j$ que satisfaça o critério de $l(j) + t_{i'} < l(j')$. A escolha de qual máquina vai ser efetivamente usada pode ser feita de forma gulosa (p. ex. aquela que mais diminui o \textit{makespan}). Mas podemos fazer essa escolha de forma aleatorizada também, escolhendo de forma aleatória que máquina vai receber o item $i'$.

\section{Problema do Corte Máximo}

Usando como base o algoritmo \lstinline{CorteMaximoGuloso} visto na \hyperref[chp:semana2]{Semana 2}, podemos aleatorizá-lo.

\begin{algorithm}
	\SetAlgoLined

	\SetKwFunction{CorteMaximoAleat}{CorteMaximoAleat}
	\Fn{\CorteMaximoAleat{$G = (V, E)$}}{
		$S \gets \emptyset$\;
		$\bar{S} \gets \emptyset$\;
		$A \gets \{s\}$ \tcc*{Vértices alcançados, existe caminho}
		\Para{$v \in A$ com maior grau}{
			\eSe{$|\left\{ (v, u) \in \delta(v): u \in S\right\}| \leq |\left\{ (v, u) \in \delta(v): u \in \bar{S}\right\}|$}{
				$S \gets S \cup \{v\}$\;
			}{
				$\bar{S} \gets \bar{S}\cup\left\{v\right\}$
			}
		}
	}
\end{algorithm}

O algoritmo usa um vértice inicial arbitrário $s$ e percorre os vértices alcançados também de maneira arbitrária (seguindo os de maior grau). Podemos escolher o vértice inicial aleatoriamente.

É importante salientar que: o número de vértices é linear dado a entrada ($|V|$), então é um número não muito grande, pode valer a pena testar para todos e pegar o melhor, fazendo uma escolha determinística.

Para juntar isso no algoritmo base \lstinline{MultiStart}, podemos fazer permutações aleatórias dos vértices e percorrê-las, usando o vértice atual da permutação como o inicial do algoritmo \lstinline{CorteMaximoAleat}. Dessa forma, ele vai continuar procurando em todos os vértices, mas de uma maneira aleatória e podemos colocar um critério de parada para não percorrer \textbf{todo} o conjunto de vértices quando ele for muito grande.

O critério de percorrer a lista de vértices alcançados também é arbitrário (guloso). Podemos também aleatorizar esse processo.

\subsection{Heurística de Busca Local}

Com relação à busca local, podemos usar como base o algoritmo da \hyperref[chp:semana3]{Semana 3}. O algoritmo \lstinline{CorteMaximoBuscaLocal} usa como base um ``corte inicial''. Esse corte inicial pode ser totalmente arbitrário, mas também pode ser construído aleatoriamente. Por exemplo, podemos sortear apenas se cada vértice está nesse corte inicial.

Já no caso desse corte inicial, o número de possibilidades de escolha é muito maior ($2^{|V|-1}-1$), não sendo viável testar todas as possibilidades.

\section{Embaralhamento de Knuth}

Normalmente precisamos percorrer listas nos algoritmos. Uma maneira de aleatorizar é percorrer a lista permutada de maneira uniforme.

Sorteamos, com probabilidade uniforme, um item do sufixo do vetor (parte do vetor não processada) e depois adicionamos o item na posição corrente.

\begin{example}
    $|1 \quad 2 \quad 3 \quad 4 \quad 5$

    Escolhemos do sufixo o número 3.

    $3 \quad |2 \quad 1 \quad 4 \quad 5$

    Escolhemos do sufixo o número 5.

    $3 \quad 5 \quad |1 \quad 4 \quad 2$

    Escolhemos do sufixo o número 1.

    $3 \quad 5 \quad 1 \quad |4 \quad 2$

    Escolhemos do sufixo o número 2.

    $3 \quad 5 \quad 1 \quad 2 \quad |4$

    Permutação final: $3 \quad 5 \quad 1 \quad 2 \quad 4$
\end{example}

O algoritmo é $O(n)$. E há a garantia de que a probabilidade é uniforme para todas as permutações possíveis.

\section{Problema do Caixeiro Viajante}

Usamos como base o algoritmo \lstinline{TSP-Guloso4} visto na \hyperref[chp:semana2]{Semana 2}.

\begin{algorithm}
    \SetAlgoLined
    \Fn{\TSPquatro{$G=(V, E), \mathrm{w}$}}{
        Seja $v$ um vértice qualquer\;
        $C \gets (v)$\;
        \Enqto{C não contém todos os vértices}{
            Seja $C = (v_1, \dots , v_i)$\;
            Escolha um vértice $x \notin C$ que \underline{minimiza} $\mathrm{w}(v_i, x)$\;
            Insira $x$ no final de $C$\;
        }
        \Retorna C
    }
\end{algorithm}

O vértice inicial $v$ pode ser escolhido aleatoriamente.

Podemos também aleatorizar o critério guloso. Por exemplo, se houver empate, podemos escolher aleatoriamente qual o vértice que será incluso no caminho.

Para aumentar a aleatoriedade, podemos ``relaxar'' o critério guloso para aumentar os empates que temos. Por exemplo, ao invés de pegar um vértice \underline{que minimiza} o caminho, podemos escolher os vértices \underline{próximos}.

Algoritmo GRASP. Se temos $x_{\min} \notin C$ mais próximo de $v_i$ e $x_{\max} \notin C$ mais distante de $v_i$. Podemos criar um critério segundo um determinado intervalo de distância: $w(v_i, x) \leq w(v_i, x_{\min}) + \alpha \left( w(v_i, x_{\max}) - w(v_i, x_{\min}) \right)$, sendo $\alpha$ um valor entre 0 e 1 que indica a ``força'' da aleatoriedade (seleciona o quão longe estão os vértices que podemos considerar).

\subsection{Heurística de Busca Local}

Também podemos aleatorizar as heurísticas de busca local, por exemplo, o algoritmo \lstinline{TSP-2-OPT}.

O circuito hamiltoniano inicial pode ser apenas uma permutação aleatória dos vértices (só funciona se o grafo for completo).

Também podemos colocar a aleatoriedade no critério de busca, ao invés de escolher arbitrariamente uma troca que traz um ganho, pode escolher aleatoriamente qual é a troca que vai ser feita.

Para fazer as permutações em um grafo não completo, podemos incluir as arestas ``que faltam'' com pesos \textbf{muito grandes}. Como o critério da busca quer encontrar trocas que diminuam o custo, ele vai acabar descartando essas arestas e ficamos com a solução do mesmo jeito. E assim, não precisamos nos preocupar em fazer um algoritmo mais complexo para gerar a permutação.
